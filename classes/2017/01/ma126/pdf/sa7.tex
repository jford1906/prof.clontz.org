\documentclass[12pt]{article}

\usepackage[letterpaper,margin=1in]{geometry}

\setlength{\parindent}{0pt}

\usepackage{amssymb}
\usepackage{amsmath}

\usepackage{multicol}

\newcommand{\assessmentTitle}{
  Standard Assessment 7
}

\usepackage{fancyhdr}
\pagestyle{fancy}
\renewcommand{\headrulewidth}{0pt}% Default \headrulewidth is 0.4pt
\renewcommand{\footrulewidth}{0pt}% Default \footrulewidth is 0pt
\chead{\footnotesize\bf\assessmentTitle}
\cfoot{\small Page \thepage}

\newcommand{\makeHeader}[4]{
\thispagestyle{empty}
\begin{center}
\fbox{\fbox{\parbox{5.5in}{\centering
#1 | #2 | #3 | #4
}}}
\end{center}
\vspace{0.1in}
\makebox[\textwidth]{
  Name:\enspace\hrulefill\hrulefill\hrulefill
}
}

\usepackage{xcolor}

\newcommand{\standardQuestion}[2]{
\newpage
\begin{center}
  \begin{tabular}{|l|c|c|}
  \hline
    \parbox{4in}{
      \textbf{#1}: This student is able to...\\
      #2
    }
  &
    \parbox{1in}{
      \vspace{0.1in}
      \footnotesize \textcolor{gray}{Mark:}
      \vspace{0.7in}

      \tiny \textcolor{gray}{(Instructor Use Only)}
    }
  &
    \parbox{1in}{
      \vspace{0.1in}
      \footnotesize \textcolor{gray}{Reattempt/ Correction:}
      \vspace{0.53in}

      \tiny \textcolor{gray}{(Instructor Use Only)}
    }
  \\\hline
  \end{tabular}
\end{center}
}

\newcommand{\csch}{\operatorname{csch}}
\newcommand{\sech}{\operatorname{sech}}



\begin{document}

\makeHeader{
MA 126 }{ Spring 2017 }{ Prof. Clontz }{ \assessmentTitle
}

\begin{itemize}
  \item Each question is prefaced with a Standard for this course.
  \item When grading, each response will be marked as follows:
  \begin{itemize}
    \item \(\checkmark\):
      The response is demonstrates complete understanding of the Standard.
    \item \(\star\):
      The response may indicate full understanding of the Standard, but
      clarification or minor corrections are required.
    \item \(\times\):
      The response does not demonstrate complete understanding of the Standard.
  \end{itemize}
  \item Only responses marked with a \(\checkmark\) mark count toward your
    grade for the semester.
    Visit the course website for more information on how to improve
    \(\star\) and \(\times\) marks.
  \item This Assessment is due after 50 minutes. All blank responses will
    be marked with \(\times\).
\end{itemize}





% \standardQuestion{C01}{
%   Derive properties of the logarithmic and exponential functions from their definitions.
% }
%
% Let \(f^\leftarrow\) denote the inverse function of an invertable function
% \(f\); in particular, if \(f(x)=\ln(x)\), then \(f^\leftarrow(x)=\exp(x)\).
%
% Use the theorem
% \(\frac{d}{dx}[f^\leftarrow(x)]=\frac{1}{f'(f^\leftarrow(x))}\) to prove that
% \(\frac{d}{dx}[\exp x]=\exp x\).
%
%
%
%
%
% \standardQuestion{C02}{
%   Prove hyperbolic function identities.
% }
%
% Use the definitions
% \[
%   \tanh(x) = \frac{e^x-e^{-x}}{e^x+e^{-x}},
%   \sech(x) = \frac{2}{e^x+e^{-x}}
% \]
% to prove the following identity.
% \[
%   1-\tanh^2(x)=\sech^2(x)
% \]



% \standardQuestion{C03}{
%   Use integration by substitution.
% }
%
% Find \(\displaystyle\int \frac{6x^2+14}{x^3+7x-3}\,dx\).



% \standardQuestion{C04}{
%   Use integration by parts.
% }
%
% Find \(\int 3x\cosh(x)\,dx\).
%
%
%
% \standardQuestion{C05}{
%   Identify and use appropriate integration techniques.
% }
%
% Draw lines matching each of the five integrals on the left with
% the most appropriate integration technique listed on the right.
% Multiple techniques may be technically possible, but choose the technique most
% useful to begin integration. Every integral and technique is used exactly
% once in the correct answer.
%
% \vspace{1em}
%
% \begin{multicols}{2}
%   \begin{itemize}
%     \item[] \(\displaystyle\int 8x^3\ln(x^4+7)\,dx\)
%     \item[] \(\displaystyle\int 8\sec^3(x)\tan^5(x)\,dx\)
%     \item[] \(\displaystyle\int \frac{1+3x}{x^3-x}\,dx\)
%     \item[] \(\displaystyle\int 2\sin(x)\cosh(x)\,dx\)
%     \item[] \(\displaystyle\int \frac{1}{\sqrt{4x^2+1}}\,dx\)
%   \end{itemize}
%   \columnbreak
%   \begin{itemize}
%     \item Integration by Substiution
%     \item Method of Partial Fractions
%     \item Trigonometric Identities
%     \item Trigonometric Substitution
%     \item Integration by Parts
%   \end{itemize}
% \end{multicols}
%
%
%
% \standardQuestion{C06}{
%   Express an area between curves as a definite integral.
% }
%
% Find a definite integral equal to the area between the curves
% \(y=3^x\) and \(y=4x+1\).
% (Do not solve your integral.)



% \standardQuestion{C07}{
%   Use the washer or cylindrical shell method to express a volume of
%   revolution as a definite integral.
% }
%
% Find a definite integral equal to the volume of
% the solid of revolution obtained by rotating
% the region bounded by \(y=|x|+1\) and \(y=2\) around the
% axis \(y=0\).
% (Do not solve your integral.)
%
%
%
%
% \standardQuestion{C08}{
%   Express the work done in a system as a definite integral.
% }
%
% Hooke's Law states that the force required to compress a spring \(x\) units
% from its natural length requires \(F(x)=kx\) units of force for some
% constant \(k\) (depending on the spring). Suppose a spring satisfies
% \(k=8\) and is naturally length \(9\). Find a definite integral equal
% to the work required to compress
% this spring from length \(7\) to length \(5\).
% (Do not solve your integral.)


\standardQuestion{C09}{
  Parametrize a curve to express an arclength or area as a definite integral.
}

Recall that \(\cosh(t)=\frac{1}{2}(e^t+e^{-t})\),
\(\sinh(t)=\frac{1}{2}(e^t-e^{-t})\), and
\(\cosh^2(t)-\sinh^2(t)=1\).

Find the arclength of \(x^2-y^2=4\) between \((2,0)\) and
\((e+\frac{1}{e},e-\frac{1}{e})\). (Hint: multiply the hyperbolic identity
by \(4\) on both sides.)

(Do not solve your integral.)


\standardQuestion{C10}{
  Use polar coordinates to express an arclength or area as a definite integral.
}

Find a definite integral equal to the circumference of the circle
\(r=3\cos\theta\).

(Do not solve your integral.)


\standardQuestion{C11}{
  Compute the limit of a convergent sequence.
}

Find \(\displaystyle\lim_{n\to\infty}\frac{4n+n^4}{5n^4+n^2-3}\).


\standardQuestion{C12}{
  Express as a limit and find the value of a convergent geometric or
  telescoping series.
}

Find the value of the convergent series
\(\sum_{k=1}^\infty 8^{-k}\).


\standardQuestion{C13}{
  Identify and use appropriate techniques for determining the convergence or
  divergence of a series.
}

Recall the following types of series and techniques for
determining series converence.

\begin{multicols}{2}
\begin{itemize}
  \item Telescoping Series
  \item Geometric Series
  \item Alternating Series Test
  \item Integral Test
  \item p-Series Test
  \item Ratio Test
  \item Root Test
  \item Comparison Test (Direct/Limit)
\end{itemize}
\end{multicols}

Label the following three series with an appropriate type of series or
technique for determining series convergence. Then label whether
each series converges or diverges (you do not need to show any work).

\begin{multicols}{3}
\[\sum_{k=0}^\infty \frac{5}{k^3}\]

\[\sum_{m=1}^\infty \frac{2}{(3m)!}\]

\[\sum_{n=3}^\infty \frac{n}{3^n+7}\]
\end{multicols}


\standardQuestion{C14}{
  Identify the domain of a function defined as a power series.
}

Prove that
\(\displaystyle
  f(x)=
  \sum_{n=0}^\infty\frac{(x-3)^n}{(n+1)!}=
  1+\frac{x-3}{2}+\frac{(x-3)^2}{6}+\frac{(x-3)^3}{24}+\dots
\)
is defined for all real numbers \(x\).


\standardQuestion{C15}{
  Generate a Taylor or Maclaurin Series from a function.
}

Generate the Maclaurin Series for \(\sin(x)\).





% \standardQuestion{S01}{
%   Find derivatives and integrals involving logrithmic and exponential functions.
% }
%
% a) Find \(\frac{d}{dz}[\ln(3e^z)]\).
%
% \vfill
%
% b) Find \(\displaystyle\int\left(2e+\frac{3}{y}\right)\,dy\).
%
% \vfill
%
%
%
% \standardQuestion{S02}{
%   Find derivatives and integrals involving hypberbolic functions.
% }
%
% a) Find \(\frac{d}{dv}[4\tanh(3v)-\sinh(v^2)]\).
%
% \vfill
%
% b) Find \(\displaystyle\int(\cosh(x)+2\sinh(x))\,dx\).
%
% \vfill



% \standardQuestion{S03}{
%   Integrate products of trigonometric functions.
% }
%
% Find \(\int2\cos^2(y)\,dy\).
%
%
%
% \standardQuestion{S04}{
%   Use trigonometric substitution.
% }
%
% Find \(\displaystyle\int\frac{z+1}{\sqrt{1-z^2}}\,dz\).
%
%
%
%
% \standardQuestion{S05}{
%   Use partial fractions to integrate rational functions.
% }
%
% a) Complete the following partial fraction expansion:
%
% \[
%   \frac{f(x)}{(x^2+3)^2(x-7)^4}
%     =
%   \frac{\hspace{3em}}{x^2+3}
%     +
%   \frac{\hspace{3em}}{(x^2+3)^2}
%     +
%   \frac{\hspace{3em}}{x-7}
%     +
%   \frac{\hspace{3em}}{(x-7)^2}
%     +
%   \frac{\hspace{3em}}{(x-7)^3}
%     +
%   \frac{\hspace{3em}}{(x-7)^4}
% \]
%
% \vspace{1em}
%
% (Assume the degree of \(f\) is less than \(8\).
% You do NOT need to solve for your constants.)
%
% \vspace{1em}
%
% b) Find \(\displaystyle\int\frac{3x^2-x+2}{(x^2+1)(x-1)}\,dx\).
%
% \vfill




% \standardQuestion{S06}{
%   Use cross-sectioning to express a volume as a definite integral.
% }
%
% Find a definite integral that equals the volume of a solid whose base
% is the triangle with vertices \((0,0)\), \((1,2)\), and \((1,-2)\),
% and whose cross-sections perpindicular to the \(x\)-axis are semicircles
% with diameters on the \(xy\) plane. (Do not solve your integral.)



%
% \standardQuestion{S07}{
%   Derive a formula for the volume of a three dimensional solid.
% }
%
% Prove that the volume of a cone with radius \(a\) and height \(h\)
% is \(V=\frac{1}{3}\pi a^2h\). (Hint: Start by letting \(y=\frac{a}{h}x\)
% be the hypotenuse of a right triangle with legs length \(h\) and \(a\).)
%
%
%
%
%
% \standardQuestion{S08}{
%   Parametrize planar curves and sketch parametrized curves.
% }
%
% a) Give a parameterization of the line segment with endpoints \((4,3)\)
% and \((-1,2)\).
%
% \vfill
%
% b) Sketch the curve parameterized by \(x=4t^2\), \(y=2t\)
% for all real numbers \(t\).
%
% \vfill



% \standardQuestion{S09}{
%   Use parametric equations to find and use tangent slopes.
% }
%
% Find the point on the parametric curve defined by \(x=t^2+3\),
% \(y=4t\) for all real numbers \(t\) that has a tangent slope of \(1\).
%
%
%
% \standardQuestion{S10}{
%   Convert and sketch polar and Cartesian coordinates and equations.
% }
%
% a) Find a Cartesian coordinate equal to the polar coordinate \(p(2,-\pi/4)\).
%
% \vfill
%
% b) Sketch the polar curve \(r=\frac{1}{\cos\theta+\sin\theta}\)
% for \(-\frac{\pi}{4}<\theta<\frac{3\pi}{4}\) in the \(xy\) plane.
%
% \vfill
%
%
%
% \standardQuestion{S11}{
%   Define and use explicit and recursive formulas for sequences.
% }
%
% Give an explicit or recursive formula matching the sequence
% \(\langle b_n\rangle_{n=0}^\infty=\langle 1,2,5,10,17,26,37,\dots\rangle \).



\standardQuestion{S12}{
  Use the alternating series test to determine series convergence.
}

Does \(\sum_{m=0}^\infty(-1)^{m+1}\frac{4+e^m}{e^{m+1}}\) converge or diverge?



\standardQuestion{S13}{
  Use the integral test to determine series convergence.
}

a) Does \(\int_0^\infty \frac{6x^2+4}{x^3+2x+5}\,dx\) converge or diverge?

\vfill
\vfill
\vfill

b) Based on (a), does \(\sum_{n=0}^\infty \frac{6n^2+4}{n^3+2n+5}\)
converge or diverge?

\vfill



\standardQuestion{S14}{
  Use the ratio and root tests to determine series convergence.
}

Does \(\sum_{n=0}^\infty\frac{n!n!}{(2n)!}\) converge or diverge?



\standardQuestion{S15}{
  Use the comparison tests to determine series convergence.
}

Does \(\sum_{n=0}^\infty\frac{2^n}{3^n+4}\) converge or diverge?





\standardQuestion{C13b}{
  Identify and use appropriate techniques for determining the convergence or
  divergence of a series.
}

Recall the following types of series and techniques for
determining series converence.

\begin{multicols}{2}
\begin{itemize}
  \item Telescoping Series
  \item Geometric Series
  \item Alternating Series Test
  \item Integral Test
  \item p-Series Test
  \item Ratio Test
  \item Root Test
  \item Comparison Test (Direct/Limit)
\end{itemize}
\end{multicols}

Label the following three series with an appropriate type of series or
technique for determining series convergence. Then label whether
each series converges or diverges (you do not need to show any work).

\begin{multicols}{3}
\[\sum_{k=0}^\infty (\frac{3}{k}-\frac{3}{k+1})\]

\[\sum_{m=1}^\infty \frac{\sin^2(m^3)+1}{m^{2/3}}\]

\[\sum_{n=0}^\infty \frac{\sqrt{n+1}}{5^n}\]
\end{multicols}


\standardQuestion{C14b}{
  Identify the domain of a function defined as a power series.
}

Find the domain of
\(\displaystyle
  f(x)=
  \sum_{n=0}^\infty(-\frac{1}{2})^n\frac{x^n}{n^2+1}=
  1-\frac{x}{4}+\frac{x^2}{20}-\frac{x^3}{80}+\dots
\). (You do not need to show your work when determining the
convergence/divergence of its endpoints.)


\standardQuestion{C15b}{
  Generate a Taylor or Maclaurin Series from a function.
}

Generate the Maclaurin Series for \(2^x\).



\newpage

\textcolor{gray}{Use this space if you need extra room for a problem:}

\end{document}
