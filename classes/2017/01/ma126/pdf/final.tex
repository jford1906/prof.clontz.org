\documentclass[12pt]{article}

\usepackage[letterpaper,margin=1in]{geometry}

\setlength{\parindent}{0pt}

\usepackage{amssymb}
\usepackage{amsmath}

\usepackage{multicol}

\newcommand{\assessmentTitle}{
  Final Exam
}

\usepackage{fancyhdr}
\pagestyle{fancy}
\renewcommand{\headrulewidth}{0pt}% Default \headrulewidth is 0.4pt
\renewcommand{\footrulewidth}{0pt}% Default \footrulewidth is 0pt
\chead{\footnotesize\bf MA 126 - Spring 2017 - Prof. Clontz - \assessmentTitle}
% \cfoot{\small Page \thepage}
\cfoot{}

\newcommand{\makeHeader}[4]{
\thispagestyle{empty}
\begin{center}
\fbox{\fbox{\parbox{5.5in}{\centering
#1 | #2 | #3 | #4
}}}
\end{center}
\vspace{0.1in}
\makebox[\textwidth]{
  Name:\enspace\hrulefill\hrulefill\hrulefill
}
}

\usepackage{xcolor}

\newcommand{\standardQuestion}[2]{
\newpage
\begin{center}
  \begin{tabular}{|l|c|}
  \hline
    \parbox{5in}{
      \textbf{#1}: This student is able to...\\
      #2
    }
  &
    \parbox{1in}{
      \vspace{0.1in}
      \footnotesize \textcolor{gray}{Mark:}
      \vspace{0.7in}

      \tiny \textcolor{gray}{(Instructor Use Only)}
    }
  \\\hline
  \end{tabular}
\end{center}
}

\newcommand{\csch}{\operatorname{csch}}
\newcommand{\sech}{\operatorname{sech}}



\begin{document}

\makeHeader{
MA 126 }{ Spring 2017 }{ Prof. Clontz }{ \assessmentTitle
}

\begin{itemize}
  \item Each question corresponds to a Standard for this course.
  \item Four questions are asked of all students, attached to this
    cover sheet.
  \item Each student may choose up to 12 additional questions from the
    provided booklet. Each choice must be clearly marked at the top
    of a provided answer sheet and stapled to this cover sheet upon
    submission.
  \item When grading, each response will be marked as follows:
  \begin{itemize}
    \item \(\checkmark\):
      The response is demonstrates complete understanding of the Standard.
    \item \(\star\):
      The response may indicate full understanding of the Standard, but
      clarification or minor corrections are required.
    \item \(\times\):
      The response does not demonstrate complete understanding of the Standard.
  \end{itemize}
  \item Up to three \(\star\) marks will be converted to \(\checkmark\) marks
    automatically.
  \item Only responses marked as \(\checkmark\) count toward your
    grade for the semester.
  \item This Assessment is due after 120 minutes.
\end{itemize}




\standardQuestion{C16a}{
  Approximate series and power series within appropriate margins of error.
}

Recall that the Maclaurin series generated by \(e^x\) is
\(\sum_{k=0}^\infty\frac{x^k}{k!}\). Use Taylor's Formula
for error \[R_n(x)=\frac{f^{n+1}(x_n)}{(n+1)!}(x-a)^{n+1}\]
to show that the value of \(\frac{1}{e}\) is within \(0.01\) of
\(\frac{3}{8}=0.375\).

\standardQuestion{C16b}{
  Approximate series and power series within appropriate margins of error.
}

Recall that the Maclaurin series generated by \(\cos(x)\) is
\(\sum_{k=0}^\infty(-1)^k\frac{x^{2k}}{{2k}!}\).
Use Taylor's Formula
for error \[R_n(x)=\frac{f^{n+1}(x_n)}{(n+1)!}(x-a)^{n+1}\]
to show that the value of \(\cos(0.5)\) is within
\(0.01\) of \(\frac{7}{8}=0.875\).




\standardQuestion{S16}{
  Find a power series converging to a function.
}

Recall that the Maclaurin series generated by \(e^x\) is
\(\sum_{k=0}^\infty\frac{x^k}{k!}\). Prove that
\(3x^2e^{x^3}=\sum_{k=0}^\infty\frac{3x^{3k+2}}{k!}\).


\standardQuestion{S17}{
  Prove the convergence of a Taylor or Maclaurin Series using Taylor’s Formula.
}

Recall that the Maclaurin series generated by \(e^x\) is
\(\sum_{k=0}^\infty\frac{x^k}{k!}\).
Use Taylor's Formula
for error \[R_n(x)=\frac{f^{n+1}(x_n)}{(n+1)!}(x-a)^{n+1}\]
to prove that
\(e^x=\sum_{k=0}^\infty\frac{x^k}{k!}\).


\newcommand{\blankStandardQuestion}{
\newpage
\begin{center}
  \begin{tabular}{|l|c|}
  \hline
    \parbox{5in}{
      \textbf{Write the code (e.g. C09b) for the question you are attempting
      to answer:}
      \vspace{0.6in}
    }
  &
    \parbox{1in}{
      \vspace{0.1in}
      \footnotesize \textcolor{gray}{Mark:}
      \vspace{0.7in}

      \tiny \textcolor{gray}{(Instructor Use Only)}
    }
  \\\hline
  \end{tabular}
\end{center}
}

\blankStandardQuestion

\newpage


\newcommand{\standardQuestionOption}[2]{
\begin{center}
  \begin{tabular}{|lc|}
  \hline
    \parbox{6in}{
      \vspace{0.1in}

      \textbf{#1}: This student is able to...\\
      #2

      \vspace{0.1in}
    }
  &
  \\\hline
  \end{tabular}
\end{center}
}



\standardQuestionOption{C01a}{
  Derive properties of the logarithmic and exponential functions from their definitions.
}

Use the definition \(\ln x = \int_1^x \frac{1}{t}\,dt\) to prove that
\(\ln(4x)=\ln x + \ln 4\) for all positive real numbers \(x\).

\standardQuestionOption{C01b}{
  Derive properties of the logarithmic and exponential functions from their definitions.
}

Let \(f^\leftarrow\) denote the inverse function of an invertable function
\(f\); in particular, if \(f(x)=\exp(x)\), then \(f^\leftarrow(x)=\ln(x)\).

Use the theorems
\(\frac{d}{dx}[f^\leftarrow(x)]=\frac{1}{f'(f^\leftarrow(x))}\) and
\(\frac{d}{dx}[\exp x]=\exp x\) to prove that
\(\frac{d}{dx}[\ln x]=\frac{1}{x}\).




\standardQuestionOption{C02a}{
  Prove hyperbolic function identities.
}

Use the definitions
\[
  \sinh(x) = \frac{e^x-e^{-x}}{2},
  \cosh(x) = \frac{e^x+e^{-x}}{2}
\]
to prove the following identity.
\[
  \cosh^2(x) = 1+\sinh^2(x)
\]




\standardQuestionOption{C02b}{
  Prove hyperbolic function identities.
}

Use the definitions
\[
  \tanh(x) = \frac{e^x-e^{-x}}{e^x+e^{-x}},
  \sech(x) = \frac{2}{e^x+e^{-x}}
\]
to prove the following identity.
\[
  \sech^2(x)+\tanh^2(x)=1
\]



\standardQuestionOption{C03a}{
  Use integration by substitution.
}

Find \(\displaystyle\int 2x^3\sqrt{x^2+1}\,dx\).

\standardQuestionOption{C03b}{
  Use integration by substitution.
}

Find \(\displaystyle\int \frac{e^{2x}}{e^{2x}+1}\,dx\).


\newpage





\standardQuestionOption{C04a}{
  Use integration by parts.
}

Find \(\int 3y^2e^y\,dy\).

\standardQuestionOption{C04b}{
  Use integration by parts.
}
assessmentTitle
Find \(\int \sin(x)\cosh(x)\,dx\).




\standardQuestionOption{C05a}{
  Identify and use appropriate integration techniques.
}

Match each of the five integrals on the left with
the most appropriate integration technique listed on the right.
Multiple techniques may be technically possible, but choose the technique most
useful to begin integration. Every integral and technique is used exactly
once in the correct answer.

\vspace{1em}

\begin{multicols}{2}
  \begin{itemize}
    \item[a)] \(\displaystyle\int \sin^3(x)\cos^4(x)\,dx\)
    \item[b)] \(\displaystyle\int \frac{x^2+x+1}{x^3+x}\,dx\)
    \item[c)] \(\displaystyle\int e^x\cos(1+e^x)\,dx\)
    \item[d)] \(\displaystyle\int x\sin(x)\,dx\)
    \item[e)] \(\displaystyle\int \frac{1}{4+x^2}\,dx\)
  \end{itemize}
  \columnbreak
  \begin{itemize}
    \item[1)] Integration by Substiution
    \item[2)] Method of Partial Fractions
    \item[3)] Trigonometric Identities
    \item[4)] Trigonometric Substitution
    \item[5)] Integration by Parts
  \end{itemize}
\end{multicols}

\newpage




\standardQuestionOption{C05b}{
  Identify and use appropriate integration techniques.
}

(See C05a for instructions.)

\begin{multicols}{2}
  \begin{itemize}
    \item[a)] \(\displaystyle\int \frac{3x+4}{(x-1)(x+2)^2}\,dx\)
    \item[b)] \(\displaystyle\int x^2\sqrt{3x^3+4}\,dx\)
    \item[c)] \(\displaystyle\int \frac{1}{4+x^2}\,dx\)
    \item[d)] \(\displaystyle\int e^x\sin(x)\,dx\)
    \item[e)] \(\displaystyle\int \sec^4(x)\tan^2(x)\,dx\)
  \end{itemize}
  \columnbreak
  \begin{itemize}
    \item[1)] Integration by Substiution
    \item[2)] Method of Partial Fractions
    \item[3)] Trigonometric Identities
    \item[4)] Trigonometric Substitution
    \item[5)] Integration by Parts
  \end{itemize}
\end{multicols}





\standardQuestionOption{C06a}{
  Express an area between curves as a definite integral.
}

Find a definite integral equal to the area between the curves
\(y=2x\) and \(y=x^2\).
(Do not solve your integral.)

\standardQuestionOption{C06b}{
  Express an area between curves as a definite integral.
}

Find a definite integral equal to the area between the curves
\(y=\sqrt{4-x^2}\) and \(y=1\).
(Do not solve your integral.)



\standardQuestionOption{C07a}{
  Use the washer or cylindrical shell method to express a volume of
  revolution as a definite integral.
}

Find a definite integral equal to the volume of
the solid of revolution obtained by rotating
the triangle with vertices \((1,1)\), \((2,1)\), and \((1,2)\) around the
axis \(x=0\).
(Do not solve your integral.)

\standardQuestionOption{C07b}{
  Use the washer or cylindrical shell method to express a volume of
  revolution as a definite integral.
}

Find a definite integral equal to the volume of
the solid of revolution obtained by rotating
the region bounded by \(y=x\) and \(y=x^2\) around the
axis \(y=0\).
(Do not solve your integral.)

\newpage





\standardQuestionOption{C08a}{
  Express the work done in a system as a definite integral.
}

Find the work required to pull up a fully extended \(50\) foot cable
that weighs \(200\) pounds.
(Do not solve your integral.)

\standardQuestionOption{C08b}{
  Express the work done in a system as a definite integral.
}

Hooke's Law states that the force required to stretch a spring \(x\) units
from its natural length requires \(F(x)=kx\) units of force for some
constant \(k\) (depending on the spring). Suppose a spring satisfies
\(k=5\) and is naturally length \(9\). Find a definite integral equal
to the work required to compress
this spring from length \(11\) to length \(14\).
(Do not solve your integral.)



\standardQuestionOption{C09a}{
  Parametrize a curve to express an arclength or area as a definite integral.
}

Find the arclength of \(x=3y^2\) between \((3,-1)\) and \((12,2)\).

(Do not solve your integral.)

\standardQuestionOption{C09b}{
  Parametrize a curve to express an arclength or area as a definite integral.
}

Recall the following.
A smooth curve parametrized by one-to-one functions
\(x(t),y(t)\) on \(a\leq t\leq b\) where
\(y(t)\geq 0\) may be rotated around the \(x\)-axis to yield a surface of
revolution. Its area is given by
\(2\pi\int_{a}^{b}y(t)\sqrt{(\frac{dx}{dt})^2+(\frac{dy}{dt})^2}\,dt\).

Use this to find
a definite integral equal to the conical surface area obtained by
rotating the line segment connecting \((1,0)\) and \((5,2)\) around the
line \(y=0\).

(Do not solve your integral.)



\standardQuestionOption{C10a}{
  Use polar coordinates to express an arclength or area as a definite integral.
}

Find a definite integral equal to the circumference of the cardioid
\(r=3+3\cos\theta\).

(Do not solve your integral.)

\standardQuestionOption{C10b}{
  Use polar coordinates to express an arclength or area as a definite integral.
}

Find a definite integral equal to the area inside the cardioid
\(r=2-2\sin\theta\).

(Do not solve your integral.)



\newpage



\standardQuestionOption{C11a}{
  Compute the limit of a convergent sequence.
}

Find \(\displaystyle\lim_{n\to\infty}\frac{4n+n^4}{5n^4+n^2-3}\).

\standardQuestionOption{C11b}{
  Compute the limit of a convergent sequence.
}

Find \(\displaystyle\lim_{n\to\infty}\frac{\sqrt{e^{2n}+1}}{e^{n+1}}\).



\standardQuestionOption{C12a}{
  Express as a limit and find the value of a convergent geometric or
  telescoping series.
}

Find the value of the convergent series
\(\sum_{k=2}^\infty \frac{2}{6^{k-1}}\).

\standardQuestionOption{C12b}{
  Express as a limit and find the value of a convergent geometric or
  telescoping series.
}

Find the value of the convergent series
\(\sum_{k=1}^\infty (\frac{2k+3}{k}-\frac{2k+5}{k+1})\).


\standardQuestionOption{C13a}{
  Identify and use appropriate techniques for determining the convergence or
  divergence of a series.
}

Recall the following types of series and techniques for
determining series converence.

\begin{multicols}{2}
\begin{itemize}
  \item Telescoping Series
  \item Geometric Series
  \item Alternating Series Test
  \item Integral Test
  \item p-Series Test
  \item Ratio Test
  \item Root Test
  \item Comparison Test (Direct/Limit)
\end{itemize}
\end{multicols}

Label the following four series with an appropriate type of series or
technique for determining series convergence. Then label whether
each series converges or diverges (you do not need to show any work).

\begin{multicols}{3}
\[1)\hspace{1em}\sum_{k=0}^\infty \frac{k}{\sqrt{k^3+1}}\]

\[2)\hspace{1em}\sum_{m=1}^\infty 2m^{-1/2}\]

\[3)\hspace{1em}\sum_{n=3}^\infty (-1)^n\frac{n}{n^2+1}\]
\end{multicols}

\newpage

\standardQuestionOption{C13b}{
  Identify and use appropriate techniques for determining the convergence or
  divergence of a series.
}

(See C13a for instructions.)

\begin{multicols}{3}
\[1)\hspace{1em}\sum_{k=0}^\infty \frac{k^2+3k+2}{2^k}\]

\[2)\hspace{1em}\sum_{m=1}^\infty \frac{4^m}{m!}\]

\[3)\hspace{1em}\sum_{n=3}^\infty \frac{2^{2n}}{5^n}\]
\end{multicols}


\standardQuestionOption{C14a}{
  Identify the domain of a function defined as a power series.
}

Prove that
\(\displaystyle
  f(x)=
  \sum_{n=0}^\infty\frac{(x-2)^n}{n!}=
  1+(x-2)+\frac{(x-2)^2}{2}+\frac{(x-2)^3}{6}+\dots
\)
is defined for all real numbers \(x\).


\standardQuestionOption{C14b}{
  Identify the domain of a function defined as a power series.
}

Prove that the domain of
\(\displaystyle
  g(x)=
  \sum_{n=1}^\infty\frac{(x-2)^n}{n2^n}=
  \frac{(x-2)}{2}+\frac{(x-2)^2}{8}+\frac{(x-2)^3}{24}+\dots
\)
is \(0\leq x<4\).


\standardQuestionOption{C15a}{
  Generate a Taylor or Maclaurin Series from a function.
}

Generate the Maclaurin Series for \(e^{x/2}\).


\standardQuestionOption{C15b}{
  Generate a Taylor or Maclaurin Series from a function.
}

Generate the Taylor Series for \(3x^2+4x+7\) at \(x=1\).
Write your answer in the form \(c_0+c_1(x-1)+c_2(x-1)^2\).





\standardQuestionOption{S01}{
  Find derivatives and integrals involving logrithmic and exponential functions.
}

a) Find \(\frac{d}{dz}[e^{2\ln(z)}]\).

b) Find \(\displaystyle\int\left(e^y-\frac{2}{y}\right)\,dy\).


\newpage


\standardQuestionOption{S02}{
  Find derivatives and integrals involving hypberbolic functions.
}

a) Find \(\frac{d}{dv}[4\sinh(3v)-\sech(v^2)]\).

b) Find \(\displaystyle\int(\sinh(x)-2\sech^2(x))\,dx\).



\standardQuestionOption{S03}{
  Integrate products of trigonometric functions.
}

Find \(\int\sin^3(y)\cos^3(y)\,dy\).



\standardQuestionOption{S04}{
  Use trigonometric substitution.
}

Find \(\displaystyle\int\frac{4}{4+z^2}\,dz\).




\standardQuestionOption{S05}{
  Use partial fractions to integrate rational functions.
}

Find \(\displaystyle\int\frac{3x^2+2x+4}{(x^2+4)(x+1)}\,dx\).




\standardQuestionOption{S06}{
  Use cross-sectioning to express a volume as a definite integral.
}

Find a definite integral that equals the volume of a solid whose base
is the triangle with vertices \((0,0)\), \((2,2)\), and \((2,0)\),
and whose cross-sections perpindicular to the \(x\)-axis are squares
with bases on the \(xy\) plane. (Do not solve your integral.)




\standardQuestionOption{S07}{
  Derive a formula for the volume of a three dimensional solid.
}

Prove that the volume of a sphere with radius \(a\)
is \(V=\frac{4}{3}\pi a^3\).





\standardQuestionOption{S08}{
  Parametrize planar curves and sketch parametrized curves.
}

a) Give a parameterization of the curve \(xy=9\) from \((1,9)\) to
\((27,\frac{1}{3})\).

b) Sketch the curve parameterized by \(x=4+t\), \(y=5-2t\)
for \(-1\leq t\leq 2\).



\newpage



\standardQuestionOption{S09}{
  Use parametric equations to find and use tangent slopes.
}

Find the slope of the tangent line to the curve defined
parametrically by \(x=4+\cos(t)\),
\(y=5+\sin(t)\) for \(0\leq t\leq 2\pi\) at the point \((4,6)\).



\standardQuestionOption{S10}{
  Convert and sketch polar and Cartesian coordinates and equations.
}

a) Find a Cartesian coordinate equal to the polar coordinate \(p(4,-2\pi/3)\).

b) Sketch the cardioid \(r=5+5\sin\theta\) in the \(xy\) plane.



\standardQuestionOption{S11}{
  Define and use explicit and recursive formulas for sequences.
}

Give an explicit or recursive formula matching the sequence
\(\langle t_n\rangle_{n=0}^\infty=\langle 0,1,3,6,10,15,21,28,\dots\rangle \).



\standardQuestionOption{S12}{
  Use the alternating series test to determine series convergence.
}

Does \(\sum_{m=0}^\infty(-1)^{m}\frac{3}{(\ln(m+5))^2}\) converge or diverge?



\standardQuestionOption{S13}{
  Use the integral test to determine series convergence.
}

a) Does \(\int_1^\infty \frac{2}{\sqrt{x}}\,dx\) converge or diverge?

b) Based on (a), does \(\sum_{n=1}^\infty \frac{2}{\sqrt{n}}\)
converge or diverge?



\standardQuestionOption{S14}{
  Use the ratio and root tests to determine series convergence.
}

Does \(\sum_{n=0}^\infty\frac{2^n}{(n+2)!}\) converge or diverge?



\standardQuestionOption{S15}{
  Use the comparison tests to determine series convergence.
}

Does \(\sum_{n=0}^\infty\frac{4n+1}{\sqrt{n^4+4}}\) converge or diverge?





\end{document}
